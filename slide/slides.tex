\documentclass{beamer}

\usepackage[french]{babel}
\usepackage[T1]{fontenc}
\usepackage[utf8]{inputenc}

\usepackage{graphicx}

\title{Lancer de rayon}
\subtitle{Première partie}
\author{Simon Chopin et Marie-Morgane Paumard}
\date{26 novembre 2013}

\usetheme{CambridgeUS}
\usecolortheme{rose}

\begin{document}

\begin{frame}
	\titlepage
\end{frame}

\section{Introduction}

\begin{frame}{Introduction}
\begin{alertblock}{Définition}
Le lancer de rayon est une technique de synthèse d'image. Il s'agit de calculer
les trajectoiress des rayons lumineux depuis la caméra jusqu'à l'objet, puis de
l'objet jusqu'à la source.
\end{alertblock}

\begin{center}
  \includegraphics[scale=0.2]{img/intro.jpg}
\end{center}

\begin{block}{Historique}
  \begin{description}
    \item[Raycasting (1968)] Lancer un rayon pixel par pixel et trouver l'objet
le plus proche.
    \item[\textsc{Whitted} (1979)] Génération des rayons : ombre, réflexion et
réfraction.
  \end{description}
\end{block}

\end{frame}

\begin{frame}{Modèles utilisés pour le rendu}
Nous avons utilisé les deux modèles suivants :
\begin{description}
  \item [Source ponctuelle unique et monochromatique] les ombres obtenues sont
très tranchées.
  \item [Sources ponctuelles polychromatiques] permet un rendu plus réaliste,
comme avec des sources secondaires.
\end{description}
\end{frame}

\begin{frame}{Table des matières}
	\tableofcontents
\end{frame}

\section{1}

\section{2}

\section{Améliorations}

\begin{frame}{Objets non sphériques}
Il est possible de remplacer les sphères par d'autres types d'objets.

\begin{block}{Exemple : création de plans}
Il suffit de créer une classe plan qui hérite de objet.
\end{block}
\end{frame}

\subsection{Gestion des matériaux}
\begin{frame}{Gestion des matériaux : la réfraction}
La réfraction est le phénomène de déviation du rayon lumineux qui se produit
lors d'un changement de milieu transparents d'indices différents : sa
trajectoire est modifiée.

\begin{center}
  \includegraphics[scale=0.2]{img/Refraction_fr.png}
\end{center}

\begin{block}{Implémentation}
Il faut munir objet de l'indice du milieu, qui permettra de calculer la
déviation du rayon, et d'un coefficient de transmission, grâce auquel on saura
quelle est l'intensité transmise.
\end{block}
\end{frame}

\begin{frame}{Gestion des matériaux : la réflexion}
La réflexion est un phénomène de déviation du rayon lumineux : en arrivant sur
une surface, une partie du rayon est renvoyé.

\begin{center}
  \includegraphics[scale=0.4]{img/introduction.png}
\end{center}

\begin{block}{Implémentation}
L'intensité du rayon réfléchi dépend du coefficient de transmission (on a
$R+T=1$). La direction de ce rayon est calculée à partir de l'angle d'incidence.
\end{block}
\end{frame}

\begin{frame}{Autres améliorations}
\begin{block}{Backend camera}
On modifie les paramètres comme la position de la caméra et du cadre, puis on enregistre la frame.

On appelle la librairie libav/ffmpeg pour créer une video à partir des images obtenues.
\end{block}
\end{frame}

\section{Conclusion}
\begin{frame}{Conclusion}
\end{frame}

\end{document}
