\documentclass{beamer}

\usepackage[french]{babel}
\usepackage[T1]{fontenc}
\usepackage[utf8]{inputenc}
\usepackage{listings}

\usepackage{graphicx}


\lstset{language=C++}

\title{Lancer de rayons}
\subtitle{Première partie}
\author{Simon Chopin et Marie-Morgane Paumard}
\date{26 novembre 2013}

\usetheme{CambridgeUS}
\usecolortheme{rose}

\begin{document}

\begin{frame}
	\titlepage
\end{frame}

\section{Introduction}

\begin{frame}{Introduction}
\begin{alertblock}{Définition}
Le lancer de rayon est une technique de synthèse d'image. Il s'agit de calculer
les trajectoires des rayons lumineux depuis la caméra jusqu'à l'objet, puis de
l'objet jusqu'à la source.
\end{alertblock}

\begin{center}
  \includegraphics[scale=0.2]{img/intro.jpg}
\end{center}

\begin{block}{Historique}
  \begin{description}
    \item[Raycasting (1968)] Lancer un rayon pixel par pixel et trouver l'objet
le plus proche.
    \item[WHITTED (1979)] Génération des rayons : ombre, réflexion et
réfraction.
  \end{description}
\end{block}

\end{frame}

\begin{frame}{Modèles utilisés pour le rendu}
Nous avons utilisé les deux modèles suivants, dans le vide :
\begin{description}
  \item [Source ponctuelle unique et monochromatique] les ombres obtenues sont
très tranchées.
  \item [Sources ponctuelles polychromatiques] permet un rendu plus réaliste,
comme avec des sources secondaires.
\end{description}
\end{frame}

\begin{frame}{Table des matières}
	\tableofcontents
\end{frame}

\section{Implémentation de base}
\begin{frame}{Diagramme UML}
    \begin{center}
        \includegraphics[scale=0.2]{img/uml.png}
    \end{center}
\end{frame}

\begin{frame}{Classe Scene}
    \begin{itemize}
        \item Point d'entrée du lanceur de rayon
        \item Ne gère que des cas de base
            \begin{itemize}
                \item Une seule source de lumière
            \end{itemize}
        \item Logique séparée en plusieurs méthodes, toutes virtuelles
        \item Pas plus de 20 lignes par méthode.
    \end{itemize}
\end{frame}

\begin{frame}[fragile]{Classe Scene — code}
    \begin{lstlisting}
class Scene {
  public:
    Scene(const LightSource & light,
          const Point & camera, double screen_width,
          const Point& screen_center);

    void add_object(Object & obj);

    virtual void render(rt::image & img);

    virtual rt::color render_ray(const Ray & ray);

    virtual Object * get_interceptor(const Ray & ray,
          std::list<Object *> * others = 0);
};
    \end{lstlisting}
\end{frame}

\begin{frame}{Classe Object}
    \begin{itemize}
        \item Abstraction d'un objet mathématique présent dans la scène
        \item Délègue la compréhension de la géométrie aux classes en héritant
        \item Utilise une classe Material annexe pour gérer les extensions dépendantes du matériau
            \begin{itemize}
                \item Transparence
                \item Réflexion
                \item \emph{et cætera}
            \end{itemize}
    \end{itemize}
\end{frame}

\begin{frame}[fragile]{Object — code}
    \begin{lstlisting}
class Object {
private:
  rt::color color;
  Material * material;
public:
  Object(const rt::color color,
      Material * material = 0);
  virtual double intersects(const Ray & r) const = 0;
  virtual rt::color render_direct(const Ray & ray,
          Scene & scene,
          const LightSource & light_source);
  virtual rt::color render_indirect(const Ray & ray,
          Scene & scene);
  virtual Impact get_impact(const Ray & ray) const = 0;
};
\end{lstlisting}
\end{frame}

\begin{frame}{Difficultés d'implémentation}
\begin{block}{Nombres à virgule flottante et plans}
    Le point d'intersection n'est que rarement exactement sur le plan.

    $\Rightarrow$ Prise en compte d'un $\epsilon$ lors du lancer du rayon vers la source.
\end{block}
\end{frame}

\section{Améliorations}


\begin{frame}{Améliorations}


\begin{center}
  \includegraphics[scale=0.2]{img/screenshot.png}
\end{center}
\end{frame}

\begin{frame}{Plusieurs sources}
  Quand on insère plusieurs sources, on effectue une moyenne quadratique sur
leurs couleurs.
  \begin{block}{Intérêts}
    \begin{itemize}
    \item Lumières polychromatiques ;
    \item Ombres plus diffuses ;
    \item Prélude aux sources secondaires.
    \end{itemize}
  \end{block}
\end{frame}

\begin{frame}{Objets non sphériques}
Il est possible de remplacer les sphères par d'autres types d'objets.

\begin{block}{Exemple : création de plans}
Il suffit de créer une classe plan qui hérite de objet.
\end{block}
\end{frame}

\subsection{Gestion des matériaux}
\begin{frame}{Gestion des matériaux : la réfraction}
La réfraction est le phénomène de déviation du rayon lumineux qui se produit
lors d'un changement de milieu transparents d'indices différents : sa
trajectoire est modifiée.

\begin{center}
  \includegraphics[scale=0.2]{img/Refraction_fr.png}
\end{center}

\begin{block}{Implémentation}
Il faut munir objet de l'indice du milieu, qui permettra de calculer la
déviation du rayon, et d'un coefficient de transmission, grâce auquel on saura
quelle est l'intensité transmise.
\end{block}

La réfraction permet aussi de se placer dans l'air ou l'eau et non pas le vide.
\end{frame}

\begin{frame}{Gestion des matériaux : la réflexion}
La réflexion est un phénomène de déviation du rayon lumineux : en arrivant sur
une surface, une partie du rayon est renvoyé.

\begin{center}
  \includegraphics[scale=0.4]{img/introduction.png}
\end{center}

\begin{block}{Implémentation}
L'intensité du rayon réfléchi dépend du coefficient de transmission (on a
$R+T=1$). La direction de ce rayon est calculée à partir de l'angle d'incidence.
\end{block}
\end{frame}

\begin{frame}{Autres améliorations}
\begin{block}{Backend camera}
On modifie les paramètres comme la position de la caméra et du cadre, puis on enregistre la frame.

On appelle la bibliothèque libav/ffmpeg pour créer une video à partir des images obtenues.
\end{block}
\end{frame}

\section{Conclusion/Questions?}
\begin{frame}{Conclusion/Questions ?}

\begin{center}
  \includegraphics[scale=0.5]{img/compiling.png}
\end{center}

\end{frame}

\end{document}
