\documentclass{beamer}
\usepackage{stmaryrd}
\usepackage{xcolor}
\usepackage[utf8]{inputenc}
\usepackage[french]{babel}
\usepackage{listings}  
\usetheme{Frankfurt}

\begin{document}

\AtBeginSection[]
{\begin{frame}{Plan}
\tableofcontents[hideothersubsections]
\end{frame}}

\title[]{Projet 3\\Lanceur de rayons}
\author[]{Simon \textsc{Chopin}, Marie-Morgane \textsc{Paumard}}
\date{\today}

\begin{frame}
\titlepage
\end{frame}



\begin{frame}{Architecture générale}
\end{frame}

\section{Algorithme de base}

\newframe{Initialisation}
\begin{itemize}
  \item On ajoute les quatre sommets d'un rectangle qui contient tous les points
  \item On commence avec une triangulation de ces quatre points
  \item[] \begin{center}\includegraphics[scale=0.1]{d2.png}\end{center}
\end{itemize}
\end{frame}

\newframe{Etape 1}
\begin{itemize}
  \item Pour ajouter un point, on cherche tous les triangles dont le cercle circonscrit contient le point. On en\`eve ces triangles et relie le point ajout\'e aux points des triangles enlev\'es.
  \item[] \begin{center}\includegraphics[scale=0.1]{d3.png}\end{center}
\end{itemize}
\end{frame}

\newframe{Etape 2}
\begin{itemize}
  \item Pour ajouter un point, on cherche tous les triangles dont le cercle circonscrit contient le point. On en\`eve ces triangles et relie le point ajout\'e aux points des triangles enlev\'es.
  \item[] \begin{center}\includegraphics[scale=0.1]{d4.png}\end{center}
\end{itemize}
\end{frame}

\newframe{R\'esultat}
\begin{itemize}
  \only<1>{\item[] \begin{center}\includegraphics[scale=0.2]{d5.png}\end{center}}
  \only<2>{\item[] \begin{center}\includegraphics[scale=0.2]{d6.png}\end{center}}
\end{itemize}
\end{frame}

\section{Matrices}
\newframe{Impl\'ementation}
%\begin{lstlisting}
type matrix = \{
  height: int;
  width: int;
  get: int $\to$ int $\to$ element;
  set: int $\to$ int $\to$ element $\to$ unit
\};;
%\end{lstlisting}
\end{frame}

\newframe{D\'ecomposition LUP}
\[
PM=\begin{pmatrix}
1 &  &  & 0\\
l_{2,1} & \ddots &  &\\
 \vdots & \ddots & \ddots &\\
l_{n,1} & \hdots & l_{n, n-1}  & 1
\end{pmatrix} \begin{pmatrix}
u_{1,1} & \hdots & u_{1,n}\\
 & \ddots & \vdots\\
0 &  & u_{n,n}
\end{pmatrix}
\]

\[
M=P^{-1}\begin{pmatrix}
1 &  &  & 0\\
l_{2,1} & \ddots &  &\\
 \vdots & \ddots & \ddots &\\
l_{n,1} & \hdots & l_{n, n-1}  & 1
\end{pmatrix} \begin{pmatrix}
u_{1,1} & \hdots & u_{1,n}\\
 & \ddots & \vdots\\
0 &  & u_{n,n}
\end{pmatrix}
\]
\end{frame}

\newframe{Calcul du d\'eterminant}
\[
\det M=\varepsilon\left(p\right) \prod\limits_{i=1}^nu_{i,i}
\]
\begin{center}
$p$ permutation associ\'ee \`a $P$\\
$\varepsilon\left(p\right)$ signature de $p$
\end{center}
\end{frame}

\section{Performances}
\newframe{Diagnostic}
    \begin{itemize}
        \item Goulet d'étranglement : recherche des triangles contenant un point ($O(n)$ avec une liste)
        \item Possibilités d'amélioration :
            \begin{itemize}
                \item Diminution du facteur constant
                \item Amélioration de la complexité asymptotique de la recherche
            \end{itemize}
    \end{itemize}
\end{frame}

\newframe{Facteur constant}
    On recalcule à chaque parcours le cercle circonscrit à tous les triangles !

    $\implies$ On peut calculer les coordonnées du centre du cercle à la construction.

    \begin{block}{Test d'inclusion}
    Test d'inclusion $\equiv$ Distance entre deux points
    \end{block}
\end{frame}

\newframe{Complexité : kd-tree}

    Un kd-tree est une généralisation des ABR à $k$ dimensions

    \begin{block}{Invariant}
        Soit $T = \left(A = \left(A_1, A_2... A_k\right), L, R, n\right), 0 \leq n < k$, on a

        \begin{itemize}
            \item $T.L.A_n \leq T.A_n < T.R.A_n$
            \item $T.L.n = T.R_n = (T.n+1) mod k$
        \end{itemize}
    \end{block}
\end{frame}

\begin{frame}{kd-tree appliqué aux triangles}
    On utilise les triangles dont le centre du cercle circonscrit est stocké.

    Chaque dimension est un extremum du cercle selon un des axes.

    Exemple :
    \begin{itemize}
        \item 0 $\to$ le maximum en abscisses
        \item 3 $\to$ le minimum en ordonnées
    \end{itemize}

    \begin{block}{Recherche}
        On peut faire une recherche par dichotomie, d'où un coût en $O(log n)$
    \end{block}
\end{frame}

\subsection{Graphes}
\begin{frame}{Comparaison des implémentations}
    \begin{figure}
        \includegraphics[scale=0.4]{complexity_comparison}
    \end{figure}
\end{frame}

\begin{frame}{Profondeur de l'arbre}
    \begin{figure}
        \includegraphics[scale=0.4]{depth_kd}
    \end{figure}
\end{frame}

\begin{frame}{Temps d'ajout d'un point}
    \begin{figure}
        \includegraphics[scale=0.4]{doc/complexity_kd}
    \end{figure}
\end{frame}


\begin{frame}{Questions}
\begin{center}Questions ?\end{center}
\end{frame}

\end{document}
